\section{Definition of the Witt algebra}

\begin{definition}[Witt algebra]
  \label{def:WittAlgebra}
  %\uses{}
  \lean{VirasoroProject.WittAlgebra, VirasoroProject.WittAlgebra.bracket}
  \leanok
  Let $\bbk$ be a field (or a commutative ring). The \term{Witt algebra} over $\bbk$ is
  the $\bbk$-vector space $\witt$ (or free $\bbk$-module) with basis $(\ell_n)_{n \in \bZ}$
  and a $\bbk$-bilinear bracket $\witt \times \witt \to \witt$ given on basis
  elements by
  \begin{align*}
    [\ell_n , \ell_m] = (n-m) \, \ell_{n+m} .
  \end{align*}
\end{definition}

With some assumptions on $\bbk$, the Witt algebra $\witt$ with the above
bracket is an $\infty$-dimensional Lie algebra over $\bbk$.

\begin{lemma}[Witt algebra is a Lie algebra]
  \label{lem:WittAlgebraIsLieAlgebra}
  \uses{def:WittAlgebra}
  \lean{VirasoroProject.WittAlgebra.instLieAlgebra}
  \leanok
  If $\bbk$ is a field of characteristic zero, then $\witt$ is
  a Lie algebra over $\bbk$.

  (In the case when $\bbk$ is a commutative ring, the $\witt$ is also
  a Lie algebra assuming the $\bbk$ has characteristic zero and
  that for $c \in \bbk$ and $X \in \witt$ we have $c \cdot X = 0$ only if
  either $c = 0$ or $X = 0$.)
\end{lemma}
\begin{proof}
  \uses{def:CyclicTripleSum}
  \leanok
  By construction, the bracket in Definition~\ref{def:WittAlgebra} is
  bilinear. It is antisymmetric on the basis vectors $\ell_n$, $n \in \bZ$,
  so by bilinearity the bracket is antisymmetric.
  It remains to check that the bracket satisfies Leibnitz rule
  (or the Jacobi identity), i.e., that for any $X, Y, X \in \witt$ we have
  \begin{align*}
    \big[X, [Y, Z] \big] = \big[[X, Y] , Z \big] + \big[Y , [X, Z] \big] .
  \end{align*}
  This formula is trilinear in $X,Y,Z$, so it suffices to verify it on
  basis vectors $X = \ell_n$, $Y = \ell_m$, $Z = \ell_k$.
  Calculating, with Definition~\ref{def:WittAlgebra}, we have
  \begin{align*}
      \big[\ell_n, [\ell_m, \ell_k] \big]
    = \big[\ell_n, (m-k) \ell_{m+k} \big]
    = (m-k) \big(n-(m+k)\big) \, \ell_{n+m+k}
  \end{align*}
  and
  \begin{align*}
         & \big[[\ell_n, \ell_m] , \ell_k \big] + \big[\ell_m , [\ell_n, \ell_k] \big] \\
    = \; & \big[(n-m) \ell_{n+m} , \ell_k \big] + \big[\ell_m , (n-k) \ell_{n+k}] \big] \\
    = \; & (n-m) \big( n+m-k \big) \, \ell_{n+m+k}  + (n-k) \big( m-(n+k) \big) \, \ell_{m+n+k} .
  \end{align*}
  Noting that
  \begin{align*}
    (n-m) \big( n+m-k \big) + (n-k) \big( m-(n+k) \big)
    %= n^2 - nk - m^2 + mk + nm - n^2 - nk - km + nk + k^2
    %= - nk - m^2 + nm + k^2
    = (m-k) \big(n-(m+k)\big) ,
  \end{align*}
  the Leibniz rule follows.
\end{proof}

\section{Virasoro cocycle}

In this section we assume that $\bbk$ is a field of characteristic zero
and $\witt$ is the Witt algebra over $\bbk$ as in
Definition~\ref{def:WittAlgebra}.

\begin{definition}[Virasoro cocycle]
  \label{def:VirasoroCocycle}
  \uses{lem:WittAlgebraIsLieAlgebra, def:LieTwoCocycle}
  \lean{VirasoroProject.WittAlgebra.virasoroCocycle}
  \leanok
  The bilinear map ${\gamma}_{\vir} \colon \witt \times \witt \to \bbk$
  given on basis elements of $\witt$ by
  \begin{align*}
    {\gamma}_{\vir}(\ell_n,\ell_m) = \frac{n^3 - n}{12} \, \delta_{n+m,0}
  \end{align*}
  is
  %a Lie algebra 2-cocycle, ${\gamma}_{\vir} \in C^2(\witt,\bbk)$.
  %We call ${\gamma}_{\vir}$
  called the \term{Virasoro cocycle}.
\end{definition}

\begin{lemma}[The Virasoro cocycle is a 2-cocycle]
  \label{lem:VirasoroCocycle_is_TwoCocycle}
  \uses{def:VirasoroCocycle}
  \lean{VirasoroProject.WittAlgebra.virasoroCocycle}
  \leanok
  The Virasoro cocycle is a 2-cocycle, ${\gamma}_{\vir} \in C^2(\witt,\bbk)$.
\end{lemma}
\begin{proof}
  % \uses{}
  \leanok
  By the construction if Definition~\ref{def:VirasoroCocycle},
  ${\gamma}_{\vir} \colon \witt \times \witt \to \bbk$ is bilinear.
  It's antisymmetry on basis elements of the Witt algebra is easily checked,
  so ${\gamma}_{\vir}$ is antisymmetric. It remains to prove the Leibniz
  rule for ${\gamma}_{\vir}$, i.e., that for $X,Y,X \in \witt$, we have
  \begin{align}\label{eq:LieTwoCocycle.leibniz}
    {\gamma}_{\vir}(X,[Y,Z]) = \; & {\gamma}_{\vir}([X,Y],Z) + {\gamma}_{\vir}(Y,[X,Z]) .
  \end{align}
  This formula is trilinear in $X,Y,Z$, so it suffices to verify it for basis
  vectors $X=\ell_n$, $Y=\ell_m$, $Z=\ell_k$. We calculate
  \begin{align}
    {\gamma}_{\vir}(\ell_n , [\ell_m , \ell_k])
    = \; & {\gamma}_{\vir}\big( \ell_n , (m-k) \ell_{m+k} \big) \\
    = \; & (m-k) \, \frac{n^3 - n}{12} \delta_{n+m+k,0} .
  \end{align}
  and
  \begin{align}
         & {\gamma}_{\vir}([\ell_n , \ell_m] , \ell_k)
          + {\gamma}_{\vir}(\ell_m , [\ell_n , \ell_k]) \\
    = \; & {\gamma}_{\vir}((n-m) \ell_{n+m} , \ell_k)
          + {\gamma}_{\vir}(\ell_m , (n-k) \ell_{n+k}) \\
    = \; & (n-m) \, \frac{(n+m)^3 - (n+m)}{12} \delta_{n+m+k,0}
          + (n-k) \, \frac{m^3-m}{12} \delta_{n+m+k,0} .
  \end{align}
  Both of the above results are nonzero only if $k=-(n+m)$,
  in which case $m-k = 2m+n$ and $n-k = 2n+m$,
  so it suffices to note that
  \begin{align*}
    (2m+n) \, (n^3 - n)
      = (n-m) \, \big( (n+m)^3 - (n+m) \big)
        + (2 n + m) (m^3 - m)
  \end{align*}
  to verify the Leibniz rule for ${\gamma}_{\vir}$.
\end{proof}

\begin{lemma}[The Virasoro cocyle is nontrivial]
  \label{lem:VirasoroCocycleNontrivial}
  \uses{lem:VirasoroCocycle_is_TwoCocycle, def:LieTwoCohomology}
  \lean{VirasoroProject.WittAlgebra.cohomologyClass_virasoroCocycle_ne_zero}
  \leanok
  The cohomology class $[{\gamma}_{\vir}] \in H^2(\witt,\bbk)$
  of the Virasoro cocycle is nonzero.
\end{lemma}
\begin{proof}
  % \uses{}
  \leanok
  Assume, by way of contradiction, that ${\gamma}_{\vir} \in B^2(\witt,\bbk)$,
  i.e., that ${\gamma}_{\vir} = \partial \beta$ for some $\beta \in C^1(\witt,\bbk)$.
  Then, in particular, for every $n \in \bZ$ we would have
  \begin{align*}
    {\gamma}_{\vir}(\ell_n,\ell_{-n})
      = \beta \big( [\ell_n, \ell_{-n}] \big) = 2 n \, \beta ( \ell_0 ) .
  \end{align*}
  By Definition~\ref{def:VirasoroCocycle}, this would imply
  \begin{align*}
    \frac{n^3-n}{12} = 2 n \, \beta ( \ell_0 )
  \end{align*}
  for all $n \in \bZ$. Considering for example $n=3$ and $n=6$, we then get
  \begin{align*}
    2 = 6 \, \beta (\ell_0)
    \qquad \text{ and } \qquad
    \frac{35}{2} = 12 \,  \beta (\ell_0) ,
  \end{align*}
  which obviously yield a contradiction.
\end{proof}

\section{Witt algebra 2-cohomology}

\begin{lemma}[Witt algebra 2-cocycle condition for basis]
  \label{lem:WittTwoCocycleEquation}
  \uses{lem:WittAlgebraIsLieAlgebra, def:LieTwoCocycle}
  \lean{VirasoroProject.WittAlgebra.add_lieTwoCocycle_apply_lgen_lgen_lgen_eq_zero}
  \leanok
  For any Witt algebra 2-cocycle $\gamma \in C^2(\witt,\bbk)$ with coefficients
  in $\bbk$, we have
  \begin{align*}
       (m-k) \, \gamma \big( \ell_n , \ell_{m+k} \big)
     + (k-n) \, \gamma \big( \ell_{m} , \ell_{n+k} \big)
     + (n-m) \, \gamma \big( \ell_k , \ell_{n+m} \big)
     \; = \; 0
  \end{align*}
  for all $n,m,k \in \bZ$.
\end{lemma}
\begin{proof}
  % \uses{}
  \leanok
  Direct calculation, using Definitions~\ref{def:WittAlgebra}
  and~\ref{def:LieTwoCocycle}.
\end{proof}

\begin{lemma}[Witt algebra 2-cocycle support assuming normalization]
  \label{lem:WittTwoCocycleSupport}
  \uses{lem:WittAlgebraIsLieAlgebra, def:LieTwoCocycle}
  \lean{VirasoroProject.WittAlgebra.lieTwoCocycle_apply_lgen_lgen_eq_zero_of_add_ne_zero}
  \leanok
  Let $\gamma \in C^2(\witt,\bbk)$ be a Witt algebra 2-cocycle such that
  $\gamma (\ell_0 , \ell_n) = 0$ for all $n \in \bZ$.
  Then for any $n,m \in \bZ$ with $n+m \ne 0$, we have
  \begin{align*}
    \gamma(\ell_n , \ell_m) = 0.
  \end{align*}
\end{lemma}
\begin{proof}
  \uses{lem:WittTwoCocycleEquation, lem:LieTwoCocycle_skew_symmetry}
  \leanok
  Apply Lemma~\ref{lem:WittTwoCocycleEquation} with $k=0$. The last term
  vanishes, and by skew-symmetry of $\gamma$, the first two terms simplify to yield
  \begin{align*}
    (m+n) \, \gamma (\ell_n , \ell_m) = 0 ,
  \end{align*}
  which, assuming $n+m \ne 0$, yields the asserted equation
  $\gamma (\ell_n , \ell_m) = 0$.
\end{proof}

\begin{lemma}[Normalization of Witt algebra 2-cocycles]
  \label{lem:WittTwoCocycleNormalization}
  \uses{def:VirasoroCocycle, def:LieTwoCoboundary}
  \lean{VirasoroProject.WittAlgebra.exists_add_bdry_eq_smul_virasoroCocycle}
  \leanok
  For any 2-cocycle $\gamma \in C^2(\witt,\bbk)$, there exists
  a coboundary $\partial \beta$ with $\beta \in C^1(\witt,\bbk)$
  such that
  \begin{align*}
    \gamma + \partial \beta \; = \; r \cdot {\gamma}_{\vir}
  \end{align*}
  for some scalar $r \in \bbk$.
\end{lemma}
\begin{proof}
  \uses{lem:WittTwoCocycleSupport, lem:WittTwoCocycleEquation}
  \leanok
  Let $\gamma \in C^2(\witt,\bbk)$ be a Witt algebra 2-cocycle.
  Define a Witt algebra 1-cocycle $\beta \in C^1(\witt,\bbk)$ by linear extension of
  \begin{align*}
    \beta (\ell_n) =
    \begin{cases}
      -\frac{1}{2} \gamma(\ell_1, \ell_{-1}) & \text{ if } n = 0 \\
      \frac{1}{n} \gamma(\ell_0, \ell_n) & \text{ if } n \ne 0 .
    \end{cases}
  \end{align*}

  For any $n \ne 0$, we calculate
  \begin{align*}
           \big( \gamma + \partial \beta \big) (\ell_0 , \ell_n)
    = \; & \gamma (\ell_0 , \ell_n) + \beta ([\ell_0,\ell_n]) \\
    = \; & \gamma (\ell_0 , \ell_n) - n \, \beta (\ell_n) \\
    = \; & \gamma (\ell_0 , \ell_n) - n \, \frac{1}{n}\gamma(\ell_0, \ell_n) = 0 .
  \end{align*}
  This property and Lemma~\ref{lem:WittTwoCocycleSupport} imply that
  \begin{align*}
    \big( \gamma + \partial \beta \big) (\ell_0 , \ell_n) = 0
  \end{align*}
  whenever $n+m \ne 0$.

  We will show the asserted equation with
  \begin{align*}
    r = 2 \, \big( \gamma + \partial \beta \big) ( \ell_2, \ell_{-2}) .
  \end{align*}

  By comparison with the Virasoro cocycle~${\gamma}_{\vir}$
  of Definition~\ref{def:VirasoroCocycle}, and using skew-symmetry,
  it remains to show that for any $n \in \bN$ we have
  \begin{align*}
    \big( \gamma + \partial \beta \big) (\ell_n , \ell_{-n})
      = r \; \frac{n^3 - n}{12} .
  \end{align*}
  The case $n = 0$ is a direct consequence of antisymmetry.
  The case $n = 1$ follows
  using the definition of $\beta$ and the calculation
  \begin{align*}
    \big( \gamma + \partial \beta \big) (\ell_1 , \ell_{-1})
    = \; & \gamma (\ell_1 , \ell_{-1}) + \beta ([\ell_1,\ell_{-1}]) \\
    = \; & \gamma (\ell_1 , \ell_{-1}) + 2 \, \beta (\ell_0) \\
    = \; & \gamma (\ell_1 , \ell_{-1}) - 2 \, \frac{1}{2} \gamma(\ell_1, \ell_{-1}) = 0 .
  \end{align*}
  The case $n = 2$ follows directly by the choice of $r$.
  We prove the equality in the cases $n \ge 3$ by induction on $n$.
  Assume the equation for smaller values of $n$.
  Apply Lemma~\ref{lem:WittTwoCocycleEquation} to $\gamma + \partial \beta$
  with $m = 1-n$ and $k = -1$ to get
  \begin{align*}
    0 = \; & (2-n) \, \big( \gamma + \partial \beta \big) (\ell_n , \ell_{-n})
           + (-1-n) \, \big( \gamma + \partial \beta \big) (\ell_{1-n} , \ell_{n-1})
           + (2 n - 1) \, \big( \gamma + \partial \beta \big) (\ell_1 , \ell_{-1}) \\
      = \; & (2-n) \, \big( \gamma + \partial \beta \big) (\ell_n , \ell_{-n})
                  - (-1-n) \, r \frac{(n-1)^3 - (n-1)}{12} \\
      = \; & (2-n) \, \big( \gamma + \partial \beta \big) (\ell_n , \ell_{-n})
           + \frac{r}{12} (n+1) n (n-1) (n-2) .
  \end{align*}
  where in the seciond step we used the induction hypothesis. Since $2-n \ne 0$, this can
  be solved for
  \begin{align*}
           \big( \gamma + \partial \beta \big) (\ell_n , \ell_{-n})
    = \; & - \frac{r}{12} \frac{(n+1) n (n-1) (n-2)}{2-n} = r \frac{n^3 - n}{12} ,
  \end{align*}
  completing the induction step.
  \end{proof}

\begin{lemma}[Witt algebra 2-cohomology is spanned by the Virasoro cocycle]
  \label{lem:WittTwoCohomologyIsOneDimensional}
  \uses{def:WittAlgebra, def:LieTwoCohomology, lem:VirasoroCocycle_is_TwoCocycle}
  \lean{VirasoroProject.WittAlgebra.rank_lieTwoCohomology_eq_one}
  \leanok
  The Lie algebra 2-cohomology $H^2(\witt,\bbk)$ of the Witt
  algebra $\witt$ with coefficients in $\bbk$ is one-dimensional
  and spanned by the class of the Virasoro cocycle ${\gamma}_{\vir}$,
  \begin{align*}
    H^2(\witt,\bbk) \; = \; \bbk \cdot [{\gamma}_{\vir}] .
  \end{align*}
\end{lemma}
\begin{proof}
  \uses{lem:WittTwoCocycleNormalization, lem:VirasoroCocycleNontrivial}
  \leanok
  This follows directly from
  Lemmas~\ref{lem:WittTwoCocycleNormalization}
  and~\ref{lem:VirasoroCocycleNontrivial}.
\end{proof}
