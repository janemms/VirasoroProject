\section{Definition of the Witt algebra}

\begin{definition}[Witt algebra]
  \label{def:WittAlgebra}
  %\uses{}
  \lean{VirasoroProject.WittAlgebra, VirasoroProject.WittAlgebra.bracket}
  \leanok
  Let $\bbk$ be a field (or a commutative ring). The \term{Witt algebra} over $\bbk$ is
  the $\bbk$-vector space $\witt$ (or free $\bbk$-module) with basis $(\ell_n)_{n \in \bZ}$
  and a $\bbk$-bilinear bracket $\witt \times \witt \to \witt$ given on basis
  elements by
  \begin{align*}
    [\ell_n , \ell_m] = (n-m) \, \ell_{n+m} .
  \end{align*}
\end{definition}

With some assumptions on $\bbk$, the Witt algebra $\witt$ with the above
bracket is an $\infty$-dimensional Lie algebra over $\bbk$.

\begin{lemma}[Witt algebra is a Lie algebra]
  \label{lem:WittAlgebraIsLieAlgebra}
  \uses{def:WittAlgebra}
  \lean{VirasoroProject.WittAlgebra.instLieAlgebra}
  \leanok
  If $\bbk$ is a field of characteristic zero, then $\witt$ is
  a Lie algebra over $\bbk$.

  (In the case when $\bbk$ is a commutative ring, the $\witt$ is also
  a Lie algebra assuming the $\bbk$ has characteristic zero and
  that for $c \in \bbk$ and $X \in \witt$ we have $c \cdot X = 0$ only if
  either $c = 0$ or $X = 0$.)
\end{lemma}
\begin{proof}
  \uses{def:CyclicTripleSum}
  \leanok
  Calculations.
\end{proof}

\section{Virasoro cocycle}

In this section we assume that $\bbk$ is a field of characteristic zero
and $\witt$ is the Witt algebra over $\bbk$ as in
Definition~\ref{def:WittAlgebra}.

\begin{definition}[Virasoro cocycle]
  \label{def:VirasoroCocycle}
  \uses{lem:WittAlgebraIsLieAlgebra, def:LieTwoCocycle}
  \lean{VirasoroProject.WittAlgebra.virasoroCocycle}
  \leanok
  The bilinear map $\gamma_\vir \colon \witt \times \witt \to \bbk$
  given on basis elements of $\witt$ by
  \begin{align*}
    \gamma_\vir(\ell_n,\ell_m) = \frac{n^3 - n}{12} \, \delta_{n+m,0}
  \end{align*}
  is a Lie algebra 2-cocycle, $\gamma_\vir \in C^2(\witt,\bbk)$.
  We call $\gamma_\vir$ the \term{Virasoro cocycle}.
\end{definition}

\begin{lemma}[The Virasoro cocyle is nontrivial]
  \label{lem:VirasoroCocycleNontrivial}
  \uses{def:VirasoroCocycle, def:LieTwoCohomology}
  \lean{VirasoroProject.WittAlgebra.cohomologyClass_virasoroCocycle_ne_zero}
  \leanok
  The cohomology class $[\gamma_\vir] \in H^2(\witt,\bbk)$
  of the Virasoro cocycle is nonzero.
\end{lemma}
\begin{proof}
  % \uses{}
  \leanok
  \ldots
\end{proof}

\section{Witt algebra 2-cohomology}

\begin{lemma}[Normalization of Witt algebra 2-cocycles]
  \label{lem:WittTwoCocycleNormalization}
  \uses{def:VirasoroCocycle, def:LieTwoCoboundary}
  \lean{VirasoroProject.WittAlgebra.exists_add_bdry_eq_smul_virasoroCocycle}
  \leanok
  For any 2-cocycle $\gamma \in C^2(\witt,\bbk)$, there exists
  a coboundary $\partial \beta$ with $\beta \in C^1(\witt,\bbk)$
  such that
  \begin{align*}
    \gamma + \partial \beta \; = \; r \cdot \gamma_{\vir}
  \end{align*}
  for some scalar $r \in \bbk$.
\end{lemma}
\begin{proof}
  % \uses{}
  \leanok
  Calculation.
\end{proof}

\begin{lemma}[Witt algebra 2-cohomology is spanned by the Virasoro cocycle]
  \label{lem:WittTwoCohomologyIsOneDimensional}
  \uses{def:WittAlgebra, def:LieTwoCohomology}
  \lean{VirasoroProject.WittAlgebra.rank_lieTwoCohomology_eq_one}
  \leanok
  The Lie algebra 2-cohomology $H^2(\witt,\bbk)$ of the Witt
  algebra $\witt$ with coefficients in $\bbk$ is one-dimensional
  and spanned by the class of the Virasoro cocycle $\gamma_\vir$,
  \begin{align*}
    H^2(\witt,\bbk) \; = \; \bbk \cdot [\gamma_{\vir}] .
  \end{align*}
\end{lemma}
\begin{proof}
  \uses{lem:WittTwoCocycleNormalization, lem:VirasoroCocycleNontrivial}
  \leanok
  This follows directly from
  Lemmas~\ref{lem:WittTwoCocycleNormalization}
  and~\ref{lem:VirasoroCocycleNontrivial}.
\end{proof}
